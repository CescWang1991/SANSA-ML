% easychair.tex,v 3.2 2012/05/15
%
% Select appropriate paper format in your document class as
% instructed by your conference organizers. Only withtimes
% and notimes can be used in proceedings created by EasyChair
%
% The available formats are 'letterpaper' and 'a4paper' with
% the former being the default if omitted as in the example
% below.
%
\documentclass{easychair}
%\documentclass[debug]{easychair}
%\documentclass[verbose]{easychair}
%\documentclass[notimes]{easychair}
%\documentclass[withtimes]{easychair}
%\documentclass[a4paper]{easychair}
%\documentclass[letterpaper]{easychair}

% This provides the \BibTeX macro
\usepackage{doc}
% \usepackage{makeidx}

% In order to save space or manage large tables or figures in a
% landcape-like text, you can use the rotating and pdflscape
% packages. Uncomment the desired from the below.
%
% \usepackage{rotating}
% \usepackage{pdflscape}

% If you plan on including some algorithm specification, we recommend
% the below package. Read more details on the custom options of the
% package documentation.
%
% \usepackage{algorithm2e}

%\makeindex

%% Document
%%
\begin{document}

%% Front Matter
%%
% Regular title as in the article class.
%
\title{Efficient Graph Kernels for RDF data using Spark}

% \titlerunning{} has to be set to either the main title or its shorter
% version for the running heads. When processed by
% EasyChair, this command is mandatory: a document without \titlerunning
% will be rejected by EasyChair

\titlerunning{Efficient Graph Kernels for RDF data using Spark}

% Authors are joined by \and. Their affiliations are given by \inst, which indexes
% into the list defined using \institute
%
\author{
Bernhard Japes\inst{1}
\and
Shinho Kang\inst{2}
}

% Institutes for affiliations are also joined by \and,
\institute{
  Informatik III, Universit\"at Bonn,
  Germany\\
  \email{bernhard.japes@uni-bonn.de}
\and
   Informatik III, Universit\"at Bonn,
   Germany\\
   \email{TODO}\\
 }
%  \authorrunning{} has to be set for the shorter version of the authors' names;
% otherwise a warning will be rendered in the running heads. When processed by
% EasyChair, this command is mandatory: a document without \authorrunning
% will be rejected by EasyChair

\authorrunning{Bernhard and Shinho}


\clearpage

%%%%%%%%%%%%%%%%%%%%%%%%%%%%%%%%%%%%%%%%%%%%%%%%%%%
\maketitle
%%%%%%%%%%%%%%%%%%%%%%%%%%%%%%%%%%%%%%%%%%%%%%%%%%%

\begin{abstract}
In this paper we study the application of graph kernels for RDF data using the popular Apache Spark\footnote{http://spark.apache.org} engine in combination with the SANSA-Stack\footnote{http://www.sansa-stack.net} data flow utilities.
We focus on an implementation of the Intersection Tree Path (ITP) Kernel, published by Gerben Klaas Dirk de Vries and Steven de Rooij in \cite{FGK}, that is based on the concept of constructing a tree for each instance and counting the number of paths in that tree.

TODO: Add further information about implementation and/or results

\end{abstract}

\setcounter{tocdepth}{2}
\pagestyle{empty}

%------------------------------------------------------------------------------
\section{Introduction}
\label{sect:Introduction}


%------------------------------------------------------------------------------
\section{Approach}
\label{sect:Approach}


%------------------------------------------------------------------------------
\section{Implementation}
\label{sect:Implementation}


%------------------------------------------------------------------------------
\section{Evaluation}
\label{sect:Evaluation}


%------------------------------------------------------------------------------
\section{Conclusion}
\label{sect:Conclusion}

\subsection{Project timeline}

\subsection{Further ideas}


%------------------------------------------------------------------------------


%------------------------------------------------------------------------------
% Refs:
%
\label{sect:bib}
\bibliographystyle{plain}
%\bibliographystyle{alpha}
%\bibliographystyle{unsrt}
%\bibliographystyle{abbrv}
\bibliography{fast_graph_kernel}

%------------------------------------------------------------------------------
% Index
%\printindex

%------------------------------------------------------------------------------
\end{document}

% EOF
